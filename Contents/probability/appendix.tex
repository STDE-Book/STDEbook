%%%%%%%%%%%%%%%%%%%%% appendix.tex %%%%%%%%%%%%%%%%%%%%%%%%%%%%%%%%%
%
% sample appendix
%
% Use this file as a template for your own input.
%
%%%%%%%%%%%%%%%%%%%%%%%% Springer-Verlag %%%%%%%%%%%%%%%%%%%%%%%%%%

\appendix
%\motto{All's well that ends well}
\chapter{Transformations of random variables}
\label{introA} % Always give a unique label
% use \chaptermark{}
% to alter or adjust the chapter heading in the running head

%%%%%%%%%%%%%%%%%%%%%%%%%%%%%%%%%%%
\section{Change of Random Variable}
\label{sec:1}
%%%%%%%%%%%%%%%%%%%%%%%%%%%%%%%%%%%

Let's consider we know the probability of a r.v. $X$, $\px$, and we now want to compute the probability density function of some variable $Y=f(X)$, that is, we need to calculate $\py$.

To understand how this new distribution or {\bf change of random variable} is calculated, let's firstly solve a particular case:

\begin{itemize}
\item $X$ is a uniform distribution in the interval $(0,1)$.

\begin{center}
\begin{tabular}{m{.5\textwidth}m{.4\textwidth}}
 \begin{equation} 
\px = \begin{cases}
1 & {\rm if} \quad 0<x<1\\
0 & {\rm otherwise} 
\end{cases} \nonumber
\end{equation} & 
\raisebox{-8ex}{\includegraphics[scale=.25]{Figures/Fig11.png}} \\ 
\end{tabular}
\end{center}

\item $Y = X^2$. Note that this change produces this transformation:
\begin{center}
\begin{tabular}{m{.5\textwidth}m{.4\textwidth}}
 \raisebox{-8ex}{\includegraphics[scale=.4]{Figures/Fig12.png}} &
 \begin{tabular}{l|l}
$x$ & $y = x^2$ \\ \hline
0.1 & 0.01\\
0.2 & 0.04\\
0.5 & 0.25\\
... & ...\\
\end{tabular} 
\end{tabular}
\end{center}

The transformation function $f(\cdot)$ is strictly increasing. So there exists its inverse function $f^{-1}(\cdot)$.

\end{itemize}
To solve this change of r.v., we are going to use the fact that:

\begin{eqnarray}
P\{0<X<0.1\} &  = & P\{0<Y<0.01\}  \nonumber\\
P\{0<X<0.2\} &  = & P\{0<Y<0.04\}  \nonumber\\
P\{0<X<0.5\} &  = & P\{0<Y<0.25\}  \nonumber
\end{eqnarray}
or, in a general case, for any value of $X$, $x_0$, we have
$$ P\{0<X<x_0\}  = P\{0<Y<y_0\}$$
where $y_0=x_0^2$ or $x_0= \sqrt{y_0}$ 

So, we can compute the cumulative distribution function of the r.v. $Y$ as
$$ F_Y(y_0) =  P\{Y<y_0\} =  P\{X<\sqrt{y_0}\}$$

Now, as the cumulative function of $Y$ is expressed in terms of the r.v $X$, we can compute it!!!
\begin{eqnarray} 
F_Y(y_0) =  P\{X<\sqrt{y_0}\} = \int_{-\infty}^{\sqrt{y_0}} \px dx = 
\begin{cases}
\int_{-\infty}^{\sqrt{y_0}} 0 dx = 0 & \quad {\rm if} \quad y_0 <0 \\[2ex]
\int_0^{\sqrt{y_0}} 1 dx = \sqrt{y_0} & \quad {\rm if} \quad 0< y_0 <1 \\[2ex]
\int_0^{1} 1 dx = 1 & \quad {\rm if} \quad y_0 >1 \\
\end{cases}  \nonumber
\end{eqnarray}

So, we have that
\begin{eqnarray} 
F_Y(y_0) =\begin{cases}
 0 & \quad {\rm if} \quad y_0 <0 \\
 \sqrt{y_0} & \quad {\rm if} \quad 0< y_0 <1 \\
 1 & \quad {\rm if} \quad y_0 >1 \\
\end{cases}  \nonumber
\end{eqnarray} 
\begin{center}
\includegraphics[scale=.4]{Figures/Fig13.png}
\end{center}
and, finally, we can obtain the density function of $Y$ as
\begin{eqnarray} 
\py = \frac{d F_Y(y)}{dy} =\begin{cases}
 \frac{1}{2\sqrt{y}} & \quad {\rm if} \quad 0<y<1 \\
 0 & \quad {\rm otherwise}  \\
\end{cases}  \nonumber
\end{eqnarray} 
\begin{center}
\includegraphics[scale=.4]{Figures/Fig14.png}
\end{center}
Now, let's try to generalize this procedure for any transformation
$$ Y = f(X) $$
being $f(\cdot)$ a strictly increasing function, so $f^{-1}(\cdot)$ exists.
\begin{enumerate}
    \item Compute the cumulative function of $Y$ (by means of $X$)
    \begin{eqnarray} 
    F_Y(y)  &= &  P\{Y<y\} =  P\{X<f^{-1}(y)\} = \int_{-\infty}^{f^{-1}(y)} \px dx = \nonumber \\ 
    & & F_X(f^{-1}(y)) - F_X(-\infty) = F_X(f^{-1}(y)) \nonumber
    \end{eqnarray} 
  Note: $F_X(-\infty) = 0$ for any cumulative distribution function  
    \item Compute the density distribution function (use the chain rule)
    \begin{eqnarray} 
    \py = \frac{d F_Y(y)}{dy} =  \frac{d F_X(f^{-1}(y))}{dy} = \frac{d F_X(x= f^{-1}(y))}{dx}  \frac{dx}{dy} = p_X(x= f^{-1}(y)) \frac{dx}{dy} \nonumber
    \end{eqnarray} 
    So, we obtain that 
    \begin{eqnarray}  \py = p_X(x= f^{-1}(y)) \frac{dx}{dy}  \nonumber \end{eqnarray} 
\end{enumerate}

This formula for the r.v. change can be generalized for any transformation function $f(\cdot)$ which is monotic (either strictly increasing or decreasing) as follows:
\begin{svgraybox}
\begin{equation} 
\py = p_X(x = f^{-1}(y))  \left| \frac{dx}{dy}\right| \label{eq:changeRV}
\end{equation} 
\end{svgraybox}

In fact, we can now use this formula over the previous example:
\begin{center}
\begin{tabular}{m{.2\textwidth}m{.5\textwidth}}
 \raisebox{-0.25ex}{$ Y = X^2$} &
 \begin{equation} 
\px = \begin{cases}
1 & {\rm if} \quad 0<x<1\\
0 & {\rm otherwise} 
\end{cases} \nonumber
\end{equation}
\end{tabular}
\end{center}
each term of the formula \eqref{eq:changeRV} is given by:
\begin{eqnarray} 
\left| \frac{dx}{dy}\right| = \left| \frac{df^{-1}(y)}{dy}\right| = \left| \frac{d\sqrt{y}}{dy}\right| = \frac{1}{2\sqrt{y}} \nonumber
\end{eqnarray} 

\begin{eqnarray} 
p_X(x = f^{-1}(y)) = p_X(x = \sqrt{y}) = \begin{cases}
1 & {\rm if} \quad 0<\sqrt{y}<1\\
0 & {\rm otherwise} 
\end{cases}  \nonumber
\end{eqnarray} 

So, we get

\begin{eqnarray} 
\py  = \frac{1}{2\sqrt{y}} p_X(x = \sqrt{y}) = \begin{cases}
\frac{1}{2\sqrt{y}} & {\rm if} \quad 0<y<1\\
0 & {\rm otherwise} 
\end{cases}  \nonumber
\end{eqnarray} 

In case the transformation function is not monotic, we have to divide the transformation into intervals where we get monotic transformations. That is, we have $Y = f(X)$ and $f(\cdot)$ is not monotic, then redefine the transformation as
\begin{eqnarray} 
Y = \begin{cases} 
f_1(X) & {\rm if } x_0 < x <x_1 \\
f_2(X) & {\rm if } x_1 < x <x_2 \\
\ldots  &  \\
f_N(X) & {\rm if } x_{N-1} < x <x_N \\
\end{cases}  \nonumber
\end{eqnarray}

where $f_1(\cdot),\ldots, f_N(\cdot) $ are monotic. Then, you can compute $\py$ as:
\begin{eqnarray} 
\py =  \sum_{n=1}^N p_X(x = f_n^{-1}(y))  \left| \frac{df_n^{-1}(y)}{dy}\right| \nonumber
\end{eqnarray} 

%%%%%%%%%%%%%%%%%%%%%%%%%%%%%%%%%%%%
\subsection{Some usual r.v. changes}

The demonstration of these changes is left as homework.

\begin{enumerate}
    \item SHIFTING of R.V. \\
    $ Y = X+a$, where $a$ is a known constant. Then,
    $$ \py =  p_X(x = y-a) $$
    when we are adding a constant to any r.v., we are shifting the distribution from the origin to the position of the constant
    
    \begin{center}
    \includegraphics[scale=.3]{Figures/Fig15.png}
    \end{center}
    
    
    \item RESCALING of R.V.\\
    $ Y = aX$, where $a$ is a known constant. Then,
    $$ \py =  \frac{1}{a} p_X(x = \frac{y}{a}) $$
    in this case we are modifying both the support of the distribution function and its height.
    
    \begin{center}
    \includegraphics[scale=.3]{Figures/Fig16.png}
    \end{center}
    
\end{enumerate}
