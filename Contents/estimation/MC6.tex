%%%%%%%%%%%%%%%%%%%%%%%%%%%%%%%%%%%%%%
\section{ML estimation of probability distributions parameters}
%%%%%%%%%%%%%%%%%%%%%%%%%%%%%%%%%%%%%%

Sometimes we may be interested in estimating the parameters of a probability distribution, such as the mean or variance of a Gaussian distribution, the decay parameter that characterizes an exponential distribution, or values $a$ and $b$ delimiting the interval in which a uniform distribution is defined.

In these cases, the prior distribution of these variables is not usually known, even more, in many cases, these parameters are said to be deterministic and are not treated as random parameters. However, if a set of observations generated from these distributions is available, we can obtain the likelihood of these variables and estimate their values with maximum likelihood criteria.

% Note that in order to use some Bayesian estimator, it would be necessary to know the posterior and without having information on the prior of these parameters we cannot know the posterior. Therefore, the only estimator we can apply in this scenario is the maximum likelihood estimator.

%%%%%%%%%%%%%%%
\begin{example}[ML estimate of the mean and variance of a one-dimensional Gaussian distribution]
\label{ex:gauss_ML}

The weight of individuals in a family of molluscs is known to obey a Gaussian distribution, but the mean and variance are unknown. The weight of $l$ individuals taken independently, $\{X^{(k)}\}_{k=1}^l$, is available.

The likelihood of the mean and the variance for a single observation $x$, in this case, is given by:
\begin{equation}
p_{X}(x) = p_{X|m,v}(x|m,v) = \frac{1}{\sqrt{2\pi v}} \exp\left[-\frac{(x-m)^2}{2 v}\right]
\end{equation}
Since we must construct the estimator based on the joint observation of $l$ observations, we will need to calculate the joint distribution of all of them which, being independent observations, is obtained as the product of individual observations:
\begin{equation}
\label{ec:conjunta_est_ML_gauss}
\begin{split}
p_{\{X^{(k)}\}|m,v}(\{x^{(k)}\}|m,v) 
	& = \prod_{k=1}^l p_{X|m,v}(x^{(k)}|m,v) \\ 
		&= \frac{1}{(2\pi v)^{l/2}} \prod_{k=1}^l \exp\left[-\frac{(x^{(k)}-m)^2}{2 v}\right]
\end{split}
\end{equation}
The ML estimators of $m$ and $v$ will be the values maximizing \eqref{ec:conjunta_est_ML_gauss}. The analytical form of this function suggests the use of the logarithm to simplify the maximization:
\begin{equation}
\label{ec:conjunta_est_ML_gauss_log}
L = \ln \left[ p_{\{X^{(k)}\}|m,v}(\{x^{(k)}\}|m,v) \right] = -\frac{l}{2} \ln(2\pi v) - \frac{1}{2v} \sum_{k=1}^l (x^{(k)}-m)^2
\end{equation}

Differentiating \eqref{ec:conjunta_est_ML_gauss_log} with respect to $m$ and $v$, we get the system of equations to solve
\begin{equation}
\begin{split}
\left.\frac{d\;L}{d\;m} \right|_{\begin{array}{l} m = \hat m_{\text{ML}} \\ v = \hat v_{\text{ML}} \end{array}} & = \left. -\frac{1}{v} \sum_{k=1}^{l} (x^{(k)} - m)\right|_{\begin{array}{l} m = \hat m_{\text{ML}} \\ v = \hat v_{\text{ML}} \end{array}}= 0\\
\left.\frac{d\;L}{d\;v} \right|_{\begin{array}{l} m = \hat m_{\text{ML}} \\ v = \hat v_{\text{ML}} \end{array}} & = \left. -\frac{l}{2v} + \frac{1}{2 v^2} \sum_{k=1}^l (x^{(k)}-m)^2 \right|_{\begin{array}{l} m = \hat m_{\text{ML}} \\ v = \hat v_{\text{ML}} \end{array}}= 0
\end{split}
\end{equation}

Solving for $m$ the first equation we get
\begin{equation}
\hat m_{\text{ML}} = \frac{1}{l} \sum_{k=1}^l x^{(k)}
\end{equation}
which is the sample average of the observations. On the other hand, we can solve the second equation for $v$ to get
\begin{equation}
\hat v_{\text{ML}} = \frac{1}{l} \sum_{k=1}^l (x^{(k)}-\hat m_{\text{ML}})^2
\end{equation}
which is the sample variance. Note that, if instead of applying the estimation function (of $m$ or $v$) on some specific observations we did it on generic values $\{X^{(k)}\}$, the estimators could be treated as random variables, i.e.,
\begin{align}
\hat M_{\text{ML}} &= \frac{1}{l} \sum_{k=1}^l X^{(k)}   \\
\hat V_{\text{ML}} &= \frac{1}{l} \sum_{k=1}^l [X^{(k)}-\hat M_{\text{ML}}]^2
\end{align}


\end{example}\vspace{0.4cm}
%%%%%%%%%%%%%
